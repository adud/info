 % !TEX TS-program = pdflatex
% !TEX encoding = UTF-8 Unicode



\documentclass[slidetop,11pt]{beamer}

\usepackage[francais]{babel}
\usepackage[T1]{fontenc}
\usepackage[utf8]{inputenc}

\usepackage{amsmath,amsfonts,amssymb,ulem,epigraph}

\usepackage[upright]{fourier}

\usepackage{shadethm}

\usepackage{subfig}

\usepackage{pgf,tikz}
\usetikzlibrary{arrows}

\usepackage{color}
\definecolor{gris_clair}{gray}{.9}
\definecolor{gris}{gray}{.35}
\definecolor{vert}{rgb}{0,0.5,0}
\definecolor{rouge}{rgb}{0.5,0,0}
\definecolor{turquoise}{rgb}{0,0.5,0.5}

%\graphicspath{{./} {./experience/utiliteordrecontrainte/}}

\usepackage{listings}           
\lstset{
language=Caml,
backgroundcolor=\color{gris_clair},
frame=single,
basicstyle=\footnotesize\ttfamily\color{gris},
identifierstyle=\color{black},
keywordstyle=\color{vert},
stringstyle=\color{rouge}, showstringspaces=false,
commentstyle=\itshape\color{turquoise},
%numbers=left, numbersep=5pt, numberstyle=\color{gris}\tiny,stepnumber=5,
breaklines=true,
literate=
  {é}{{\'e}}1 {É}{{\'E}}1 {à}{{\`a}}1 {è}{{\`e}}1% 
  {À}{{\`A}}1 {È}{{\'E}}1 {ë}{{\"e}}1 {ï}{{\"i}}1%
  {â}{{\^a}}1 {ê}{{\^e}}1 {î}{{\^i}}1 {ô}{{\^o}}1% 
  {û}{{\^u}}1 {Â}{{\^A}}1 {Ê}{{\^E}}1 {Î}{{\^I}}1%
  {Ô}{{\^O}}1 {œ}{{\oe}}1 {Œ}{{\OE}}1 {æ}{{\ae}}1%
  {Æ}{{\AE}}1 {ç}{{\c c}}1 {Ç}{{\c C}}1 {€}{{\EUR}}1%
  }

\usetheme{Warsaw}
\usecolortheme{beaver}

\graphicspath{{./} {./../source/}}

\title{Autour du Manège Enchanté}
\subtitle{intersections routières}
\date{}
\author{Antonin Dudermel}

\begin{document}

\newshadetheorem{defin}{Définition}
\newshadetheorem{theo}{Théorème}

\frame{\titlepage}

\begin{frame}
	\epigraph{Tournicoti, tournicotin}{\it Zébulon}
\end{frame}

\begin{frame}
	\tableofcontents
\end{frame}

\section{Un Graphe pour le manège}
	\subsection{Le Manège}
\begin{frame}
	\frametitle{Le Manège enchanté}
	\begin{center}
		\includegraphics[scale=0.4]{./images/magic-brit}
	\end{center}
\end{frame}

%\begin{frame}
%	\frametitle{Fonctionnement}
%	\begin{itemize}
%		\item 1 grand rond-point central tournant dans le sens inverse
%		\item 5 petits ronds-points latéraux
%		\item Des lignes de "cédez le passage" avec de l'espace pour plusieurs voitures
%	\end{itemize}
%\end{frame}

\begin{frame}
	\frametitle{Comment circuler ?}
	\begin{figure}
		\begin{center}
			\includegraphics[scale=0.3]{./images/itin}
			\caption{itinéraires possibles}
		\end{center}
	\end{figure}
\end{frame}


	\subsection{Modéliser par un graphe}
\begin{frame}
	\frametitle{Graphe du rond-point}
	\begin{figure}
		\begin{center}
			\subfloat[le manège]{
				\includegraphics[scale=0.2]{../manege-graphe}
				}
			\subfloat[le rond-point]{
				\includegraphics[scale=0.2]{../rp-graphe}}
		\end{center}
		\caption{un graphe adapté au manège}
	\end{figure}
\end{frame}

	\subsection{Diminution des distances}

%\begin{frame}
%
%	\frametitle{chemins plus courts}
%	Objectif : appliquer l'algorithme de Floyd-Warshall pour montrer une diminution du trajet	\par 
%	Réalisation : création du graphe pour le manège enchanté et un rond-point classique, et comparaison des distances entrées-sorties pour les deux graphes
%	
%\end{frame}

\begin{frame}
	\frametitle{Résultats}
\begin{figure}
	\caption{Rapport entre la distance entrée-entrée pour le rond-point et le manège (en \%)}
	\begin{center}
	\begin{tabular}{ r|c c c c c}
   entrée & 0 & 4 & 8 & 12 & 16 \\ \hline
0 & 21 & 100 & 100 & 60 & 36 \\
4 & 36 & 21 & 100 & 100 & 60 \\
8 & 60 & 36 & 21 & 100 & 100 \\
12 & 100 & 60 & 36 & 21 & 100 \\
16 & 100 & 100 & 60 & 36 & 21 \\
	\end{tabular}
	\end{center}
\end{figure}
\end{frame}

	\subsection{Résistance aux accidents}

\begin{frame}
	\frametitle{résistant aux accidents ?}
	Supposons qu'il y ait un accident sur une section, le rond-point est-il toujours fonctionnel ? Plus formellement, si $G=(V,E)$ est un graphe orienté fortement connexe, $(a,b) \in E $, $G^\prime = (V,E\backslash \{ (a,b) \})$ est-il fortement connexe ?
\end{frame}

\begin{frame}
	\begin{theorem}
		Soit $G=(V,E)$ un graphe orienté fortement connexe, $(a,b) \in E$, alors :
		$G^\prime = (V,E\backslash \{ (a,b) \})$ est fortement connexe SSI il existe un chemin de $a$ à $b$
		dans $G^\prime$
	\end{theorem}
	Résultat : les seules sections dont un accident bloque le manège sont les sections d'introduction dans le manège (contre toutes pour le rond-point classique)
\end{frame}

\section{Un Modèle par automate cellulaire}
	\begin{frame}
		\frametitle{Un Modèle par automate cellulaire}
		\begin{center}
			\includegraphics[scale=0.42]{./images/localdebut}\vfill
			\includegraphics[scale=0.42]{./images/localsature}
		\end{center}
	\end{frame}

	\subsection{Automate cellulaire}
%		\begin{frame}
%	\frametitle{Modèle Nagel-Schreckenberg (NaSch)}
%	\begin{enumerate}
%		\item Accélération 
%		\item Décélération
%		\item Facteur aléatoire
%		\item Mouvement
%	\end{enumerate}
%\end{frame}


	\subsection{Le problème des intersections}
%\begin{frame}
%\frametitle{adapter le modèle}
%	Deux types d'objets : les sections et les intersections. Il faut déterminer les comportements aux intersections
%\end{frame}	

\begin{frame}
	\frametitle{un premier modèle : priorité absolue}
	\begin{figure}
		\begin{center}
			\includegraphics[scale=0.5]{./images/localdebut}
		\end{center}
		\caption{Un exemple d'intersection 2-1}
	\end{figure}
	
	Si une voiture de la voie prioritaire décide de passer, la voiture non-prioritaire la laisse passer : on indente la première mais pas la deuxième.
\end{frame}	
	
	
\begin{frame}
	\frametitle{Modèle de Chen Rui-Xiong, Bai Ke-Zhao, Liu Mu-Ren}
	Idée : anticiper le mouvement des deux voitures souhaitant passer, puis indenter les voitures l'une après l'autre
		\begin{equation}
			t_i = \frac{x_I - x_i}{\min(v_{max},d_i-1,v_i+1)}
		\end{equation}
\end{frame}
	\subsection{Étude locale : comparer les modèles sur une intersection}

\begin{frame}
	\frametitle{Étude locale}
	Expérience simpliste : deux entrées et une sortie \\
	Objectifs
	\begin{enumerate}
		\item Étudier les modèles dans un cas simple
	 	\item Chercher les limites du rond-point
		\item Déterminer les différences entre les deux modèles
	\end{enumerate}
	\begin{figure}
		\begin{center}
			\includegraphics[scale=0.5]{./images/localdebut}	
		\end{center}
		\caption{L'expérience en cours}
	\end{figure}		
\end{frame}	

\begin{frame}
	\frametitle{Méthode d'étude : le diagramme fondamental}
	Grandeurs étudiées : 
	\begin{itemize}
		\item flux $J \text{ en } \mathrm{veh}/\mathrm{s}$
		\item vitesse $v \text{ en } \mathrm{m}/\mathrm{s}$
		\item densité $\rho \text{ en } \mathrm{veh}/\mathrm{m}$
	\end{itemize}
	$$J = \rho<v>$$
\end{frame}

\begin{frame}
	\frametitle{Diagramme fondamental}
	$$J = f(\rho)$$
	\begin{figure}
		\begin{center}	
			\includegraphics[scale = 0.6]{./images/dfondcomp}
		\end{center}
	\caption{Un exemple de diagramme fondamental}
	\end{figure}
\end{frame}
	
%\begin{frame}
%	\frametitle{Expérience :}
%	Sur la voie non-prioritaire : un flot continu et peu dense de voitures.\\
%	Sur la voie prioritaire : un flot croissant de voitures.
%\end{frame}
	
\begin{frame}
	\frametitle{Premiers résultats}
	\begin{figure}
		\begin{center}
			\includegraphics[scale=0.4]{./diagrammes-fondamentaux/localabsdyn0w}
		\end{center}
	\caption{Comparaison absolu-dynamique}
	\end{figure}
\end{frame}

\begin{frame}
	\frametitle{Interprétation}
	Informations : 
		\begin{itemize}
			\item Met en évidence le problème de ces intersections
			\item On ne voit pas une grande différence entre les deux modèles		
		\end{itemize}
		
	Améliorations possibles :
		\begin{itemize}
			\item Étudier un champ plus réduit
			\item Modifier le modèle
		\end{itemize}

\end{frame}

%\begin{frame}
%	\frametitle{Champ réduit}
%		\begin{figure}
%		\begin{center}
%			\includegraphics[scale=0.4]{./diagrammes-fondamentaux/localabsdyn0wred}
%		\end{center}
%	\caption{Comparaison sur un champ réduit}
%	\end{figure}
%\end{frame}

\begin{frame}
	 \frametitle{Améliorer le modèle}
	 Ajout d'un temps de réaction pour un redémarrage.
	 \begin{figure}
	 	\begin{center}
	 		\includegraphics[scale=0.4]{./diagrammes-fondamentaux/localabs0w1w}
	 	\end{center}
	 	\caption{Étude locale avec temps de démarrage}
	 \end{figure}
\end{frame}

\begin{frame}
	\frametitle{Résultats}
	Observations :
		\begin{itemize}
			 \item Blocage à des densités plus élevées
			 \item Une différence entre les modèles
		\end{itemize}
	Explications :
		\begin{itemize}
			\item saturation de la voie principale
			\item augmentation de l'importance de l’arrêt
		\end{itemize}
	\begin{figure}
		\begin{center}
			\includegraphics[scale=0.4]{./images/localsature}
		\end{center}
		\caption{début de saturation}
	\end{figure}
\end{frame}
	

	\subsection{Limites du modèle pour le manège}
	
\begin{frame}
	\frametitle{Difficultés à modéliser}
	\begin{itemize}
		\item de nombreux objets
		\item routes de tailles variables : besoin de connaître les fréquences de chaque route
		\item le modèle ne tient pas compte des multiples entrées
	\end{itemize}
\end{frame}

\section*{annexe}

\begin{frame}
	\includegraphics[scale=3]{./images/highway-engineers-pranks-hz}
\end{frame}

\begin{frame}
\frametitle{validité du modèle Nagel-Schreckenberg}
\includegraphics[scale = 0.7]{./images/dfondcomp}
\includegraphics[scale = 0.7]{./images/dfondcompre}
\end{frame}

\begin{frame}
	\frametitle{Comparaison de divers temps de réaction}
	\begin{figure}
		\begin{center}
			\subfloat[temps 1]{
				\includegraphics[scale=0.35]{./diagrammes-fondamentaux/localabsdyn1w}
			}
			\subfloat[temps 2]{
				\includegraphics[scale=0.35]{./diagrammes-fondamentaux/localabsdyn2w}
			}
		\end{center}
	\end{figure}
\end{frame}


\end{document}
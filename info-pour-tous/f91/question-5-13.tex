% !TEX TS-program = pdflatex
% !TEX encoding = UTF-8 Unicode



\documentclass[11pt,a4paper]{article}

\usepackage[francais]{babel}
\usepackage[T1]{fontenc}
\usepackage[utf8]{inputenc}

\usepackage{amsmath,amsfonts,amssymb}

\usepackage{geometry}
\geometry{margin=75pt}

\usepackage[upright]{fourier}

\usepackage{shadethm}

\usepackage{color}
\definecolor{gris_clair}{gray}{.9}
\definecolor{gris}{gray}{.35}
\definecolor{vert}{rgb}{0,0.5,0}
\definecolor{rouge}{rgb}{0.5,0,0}
\definecolor{turquoise}{rgb}{0,0.5,0.5}

\usepackage{listings}
\usepackage{paralist}
\usepackage{stmaryrd}

\lstset{
language=Python,
backgroundcolor=\color{gris_clair},
frame=single,
basicstyle=\footnotesize\ttfamily\color{gris},
identifierstyle=\color{black},
keywordstyle=\color{vert},
stringstyle=\color{rouge}, showstringspaces=false,
commentstyle=\itshape\color{turquoise},
%numbers=left, numbersep=5pt, numberstyle=\color{gris}\tiny,stepnumber=5,
breaklines=true,
literate=
  {é}{{\'e}}1 {É}{{\'E}}1 {à}{{\`a}}1 {è}{{\`e}}1% 
  {À}{{\`A}}1 {È}{{\'E}}1 {ë}{{\"e}}1 {ï}{{\"i}}1%
  {â}{{\^a}}1 {ê}{{\^e}}1 {î}{{\^i}}1 {ô}{{\^o}}1% 
  {û}{{\^u}}1 {Â}{{\^A}}1 {Ê}{{\^E}}1 {Î}{{\^I}}1%
  {Ô}{{\^O}}1 {œ}{{\oe}}1 {Œ}{{\OE}}1 {æ}{{\ae}}1%
  {Æ}{{\AE}}1 {ç}{{\c c}}1 {Ç}{{\c C}}1 {€}{{\EUR}}1 ,
morekeywords={len,input,range}}         


\title{Correction de l'exercice 5.13}
\date{}
\author{Antonin Dudermel \and Ambroise Poulet \and Mathias Goffette}

\begin{document}

\newshadetheorem{defin}{Définition}
\newshadetheorem{theo}{Théorème}

\maketitle

\begin{it}
Il serait intéressant (avant de lire la correction) de calculer \;$\mathtt{f91}$ pour quelques valeurs non-triviales et de se poser la question subsidiaire : pourquoi cette fonction s'appelle-t-elle \;$\mathtt{f91}$ ?
\end{it}
\par
On propose de démontrer la terminaison de la fonction suivante :

\begin{lstlisting}
def f91(n):
    if n <= 100:
        return f91(f91(n + 11))
    else:
        return n - 10
\end{lstlisting}

\par
Le cas des entiers strictement supérieurs à $100$ ne pose aucun problème, reste à démontrer la terminaison pour des entiers inférieurs à $100$.
Faisons-le en démontrant un résultat plus fort : $\forall n \in \mathbb{Z}, n\leq 101$ soit $P_n$ la propriété : \og{} $\mathtt{f91(n)}$ termine et renvoie $91$ \fg{}
\par
Par récurrence forte descendante :\\
\begin{description}
\item[Initialisation] \quad On a directement $P_{101}$
\item[Hérédité] \quad Soit $n\in \mathbb{Z}, n \leq 101$ tel que pour tout $ k \in \llbracket n+1;101\rrbracket$ on ait $P_{k}$; par disjonction de cas : 
\begin{itemize}[\textbullet]
	\item {\itshape Si $n\geqslant 91$ } alors $\mathtt{f91(n)}$ calcule et retourne, si possible $\mathtt{f91(f91(n+11))}$. Or $n+11 > 100$. Ainsi $\mathtt{f91(n+11)}$ renvoie $n+11-10$ soit $n+1$. Par hypothèse de récurrence : $\mathtt{f91(n+1)}$ termine et renvoie 91. Donc $\mathtt{f91(n)}$ termine et renvoie $91$.
	\item {\itshape Sinon, $n < 91$}, alors $\mathtt{f91(n)}$ calcule et retourne si possible $\mathtt{f91(f91(n+11))}$. Or $n+11\in \llbracket n+1;101 \rrbracket$ donc $\mathtt{f91(n+11)}$ termine et retourne $91$. Or $91 \in \llbracket n+1;101 \rrbracket$, donc $\mathtt{f91(f91(n+11))}$ termine et retourne $91$, donc $\mathtt{f91(n)}$ termine et retourne $91$.
\end{itemize}
\end{description}
Par récurrence : $\forall n \in \mathbb{Z}, n \leq 101$, $\mathtt{f91(n)}$ termine,
donc $\forall n \in \mathbb{Z}$ , $\mathtt{f91(n)}$ termine.

\end{document}